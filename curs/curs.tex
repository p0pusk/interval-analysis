\documentclass[12pt]{article}

\usepackage{graphicx} % Required for the inclusion of mages
\usepackage[square,sort,comma,numbers]{natbib}
\usepackage{amsmath} % Required for some math elements 
\usepackage{import}
\usepackage[section]{placeins}
\usepackage{subfiles}

\usepackage{cmap}
\usepackage[T2A]{fontenc}
\usepackage[utf8x]{inputenc}
\usepackage[english, russian]{babel}
\usepackage[a4paper,left=15mm,right=15mm,top=30mm,bottom=20mm]{geometry}

\begin{document}
  
  \begin{titlepage}
        \begin{center}
            \large
            Санкт-Петербургский политехнический университет\\Петра Великого\\
            \vspace{0.5cm}
            Институт прикладной математики и механики\\
            \vspace{0.25cm}
            Кафедра «Прикладная математика»
            \vfill
            \textsc{\LARGE\textbf{Отчет по курсовой работе}}\\[5mm]
            \Large
            по дисциплине\\"Интервальный анализ" \\
            Тема: "Нечеткие множества - система уравнений"
        \end{center}
        \vfill
        \begin{tabular}{l p{175pt} l}
            Выполнил студент \\ группы 5030102/00201 && Сон Артём Игоревич
            \vspace{0.25cm}
            \\Проверил \\ доцент, к.ф.-м.н. && Баженов Александр Николаевич
        \end{tabular}
        \vfill
        \begin{center}
            Санкт-Петербург \\ 2024 г.
        \end{center}


\end{titlepage}

\section{Постановка задачи}

Решить полностью нечеткую систему линейных уравнений $\tilde{A} \times \tilde{x} = \tilde{b}$ c 
$\tilde{A} = (A, M, N) > \tilde{0}$, матрица $A$ невырожденная, $\tilde{b} = (b,
g, h) > \tilde{0}$ 



\section{Теория}

\begin{enumerate}
  \item Треугольное нечеткое чесло $ \bar{A} = (m, \alpha, \beta) $ задается
    функцией принадлежности: 


    $$
    \mu_{\bar{A}} (x) = \begin{cases}
      1 - \frac{m - x}{\alpha}, & m - \alpha \leq x < m, \alpha > 0  \\
      1 - \frac{x - m}{\beta}, & m \leq x \leq m + \beta, \beta > 0  \\
      0, & \text{во всех остальных случаях} \\
    \end{cases}
    $$

    Число $\tilde{0} = (0, 0, 0)$  считается нулевым треугольным нечетким
    числом.

    % рисунок

  \item 
    Нечеткое число $\bar{A}$ называется положительным ($\bar{A} > 0$), если его
    функция принадлежности $\mu_{\bar{A}}(x) = 0, \forall x \leq 0$, и
    отрицательным ($\bar{A} < 0$), если его функция принадлежности
    $\mu{\bar{A}}(x) = 0, \forall x \geq 0$.

    Число $\bar{A} = (m, \alpha, \beta)$ является положительным, если $m -
    \alpha \geq 0$.

  \item
    Два нечетких треугольных числа $\bar{M} = (m, \alpha, \beta)$ и $\bar{N} =
    (n, \gamma, \delta)$ равны тогда и только тогда, когда 
    $m = n, \alpha = \gamma, \beta = \delta$.

  \item
    Суммой двух нечетких треугольных чисел $\bar{M} = (m, \alpha, \beta)$ и 
    $\bar{N} = (n, \gamma, \delta)$ называется число
    $ (m, \alpha, \beta) \oplus (n, \gamma, \delta) = (m + n, \alpha + \gamma,
    \beta + \delta) $

  \item
    Произведение двух положительных нечетких чисел $\bar{M} = (m, \alpha, \beta)
    > 0$ и $ \bar{N} = (n, \gamma, \delta) > 0 $ при малых значениях $\alpha,
    \beta$ по сравнению с $m$ и малых $\gamma, \delta$ по сравнению с $n$
    приближенно определяется, как нечеткое число вида

    $ (m, \alpha, \beta) \otimes (n, \gamma, \delta) = (mn, m\gamma + n \alpha,
    m\delta + n \beta)$.

    В частном случае умножения на четкое число $\lambda$:

    $$
    \lambda \otimes (m, \alpha, \beta) = \begin{cases}
      (\lambda m, \lambda \alpha, \lambda \beta), & \lambda > 0, \\
      (\lambda m, - \lambda \beta, - \lambda \alpha), & \lambda < 0 \\
    \end{cases}
    $$

    Когда разброс, характеризуемый $\alpha, \beta, \gamma, \delta$ не является
    малым, может быть использована более точная формула умножения:

    $ (m, \alpha, \beta) \otimes (n, \gamma, \delta) = (mn, m \gamma + n \alpha -
    \alpha \gamma, m \delta + n \beta - \beta \delta )$.

  \item
    Матрица $\tilde{A} = (\tilde{a_{ij}})$ называется нечеткой, если каждый ее
    элемент представляется нечетким числом.

    Нечеткая матрица называется положительной ($\tilde(A) > 0$), если каждый ее
    элемент положителен. Аналогично определяется неотрицательные, отрицательные,
    неположительные нечеткие матрицы.

    Нечеткая матрица может быть представлена в форме $\tilde{A} =
    (\tilde{a_{ij}}) = ((a_{ij}, \alpha_{ij}, \beta_{ij}))$, или
    $\tilde{A} = (A, M, N)$, где $A = (a_{ij}), M = (\alpha_{ij}), N =
    (\beta_{ij})$, три матрицы с четкими элементами.

  \item
    Система вида

    $$
    \tilde{A} \otimes \tilde{x} = \tilde{b}
    $$

    с нечеткой прямоугольной матрицей $\tilde{A} = (\tilde{a_{ij}}, i = 1,...,n$
    и нечеткой матрицей $\tilde{b} = (\tilde{b_{ij}})$ размером $m \times 1$
    называется полностью нечеткой линейной системой.

    В расширенной форме ее можно переписать в виде:

    \begin{center}

    $(\tilde{a_{11}} \otimes \tilde{x_1}) \oplus ... \oplus (\tilde{a_{1n}}
    \otimes \tilde{x_n}) = \tilde{b_1}$,

    $(\tilde{a_{21}} \otimes \tilde{x_2}) \oplus ... \oplus (\tilde{a_{2n}}
    \otimes \tilde{x_n}) = \tilde{b_2}$, 

    ...

    $(\tilde{a_{n1}} \otimes \tilde{x_n}) \oplus ... \oplus (\tilde{a_{nn}}
    \otimes \tilde{x_n}) = \tilde{b_n}$.
    \end{center}

\end{enumerate}

\subsection{Решение полностью нечетких СЛАУ}

Метод нахождения положительного решения $\tilde{x} = (x, y, z) > \tilde{0}$
полностью нечеткой линейной системы $\tilde{A} \times \tilde{x} = \tilde{b}$ c 
$\tilde{A} = (A, M, N) > \tilde{0}$, матрица $A$ невырожденная, $\tilde{b} = (b,
g, h) > \tilde{0}$ при выполнении предположений о малых значениях $\alpha,
\beta$ по сравнению с $m$ и малых $\gamma, \delta$ по сравнению с $n$:

\begin{center}
  $x = A^{-1}b, $ \\
  $y = A^{-1}(g - Mx),$ \\
  $z = A^{-1} (h - Nx).$ \\
\end{center}

Условия, при которых полностью нечеткая СЛАУ имеет положительное решение:

Если $\tilde{A}$ - неотрицательная нечеткая матрица, $\tilde{x} $ -
неотрицательный нечеткий вектор, $\tilde{b}$ - известный нечеткий вектор.

\par

Решение системы $(A, M, N) \otimes (x, y, z) = (b, g, h)$ в случае, когда
матрица $A$ прямоугольная и не делается предположений о малых значениях $\alpha,
\beta$ по сравнению с $m$, и малых $\gamma, \delta$ по сравнению с $n$.

В этом случае применим более четкую формулу умножения нечетких чисел вида

$ (m, \alpha, \beta) \otimes (n, \gamma, \delta) = (mn, m \gamma + n \alpha -
\alpha \gamma, m \delta + n \beta - \beta \delta )$:

$(Ax, Ay + Mx - My, Az + Nx + Nz) = (b, g, h)$.

Применяя условия равенства нечетких чисел получаем

\begin{center}
$ Ax = b $, \\
$ Ay + Mx - My = g $ \\
$ Az + Nx + Nz = h $ \\
\end{center}

Отсюда

\begin{center}
  $ x = A^{-1}b $, \\
  $ y = (A - M)^{-1} (g - Mx) $, \\
  $ z = (A + N)^{-1} (h - Nx) $.
\end{center}

или

\begin{center}
$ x = A^{-1}b, y = (A - M)^{-1} (g - MA^{-1}b), z = (A + N)^{-1} (h - NA^{-1}b) $.
\end{center}


\subsection{Выводы}
Рассмотрена теория полных нечетких систем линейных уравнений, и способы их
решения. Также была написана программа на языке python, для их решения.

\end{document}
